\chapter{Marco teorico}
\label{ch:marco_teorico}
\graphicspath{{figs/}}



% -------------------------------------------------------------
% SECCIÓN 1: Fundamentos del Problema
% -------------------------------------------------------------
\section{Sistemas de Tiempo Real y Trazabilidad}
\label{sec:sistemas_tiempo_real}

Este apartado establece el contexto y la necesidad de la investigación, justificando por qué la medición temporal es crítica en el campo de aplicación. Este apartado lo voy a llenar con Charlie, dado que tambien quiero tener mas información de como fue impulsado el trabajo.

\subsection{Definición de Sistemas de Tiempo Real (STR)}
\label{ssec:definicion_str}
% Definición, tipos (hard, soft), y la importancia del determinismo.

\subsection{El Problema de la Medición de Latencia (Intrusión)}
\label{ssec:problema_latencia}
% Explicar por qué los métodos de software son insuficientes (efecto de intrusión) y la necesidad de soluciones de bajo nivel.

% -------------------------------------------------------------
% SECCIÓN 2: Plataforma de Hardware: FPGA y Arquitectura Zynq
% -------------------------------------------------------------
\section{Arquitectura Heterogénea en Sistemas de Lógica Programable}
\label{sec:arquitectura_fpga}

Se describe la plataforma física sobre la que se implementará la solución, detallando la interconexión PS-PL.

\subsection{Fundamentos de Field-Programmable Gate Arrays (FPGA)}
\label{ssec:fundamentos_fpga}
% Brevemente la tecnología: Lógica Programable (PL) vs. Processing System (PS).

\subsection{Arquitectura System-on-Chip (SoC) Zynq}
\label{ssec:arquitectura_zynq}
% Detalle de la interconexión PS-PL, el ARM embebido y su rol como plataforma anfitriona.
% Incluir la Figura 1 (esquema general de la placa) si no se incluyó en la Introducción.

\subsection{Uso de Linux en el Processing System (PS)}
\label{ssec:linux_ps}
% Justificación del uso de Linux para recepción, procesamiento y almacenamiento de datos.

% -------------------------------------------------------------
% SECCIÓN 3: La Unidad Bajo Prueba y la Interconexión
% -------------------------------------------------------------
\section{Procesadores Soft-Core y Protocolo de Comunicación}
\label{sec:softcore_comunicacion}

Se centra en la unidad que genera los eventos y el medio por el cual se transfieren.

\subsection{Procesador Soft-Core RISC-V}
\label{ssec:riscv}
% Introducción al ISA RISC-V.
% Requisitos para la síntesis en PL: acceso a registros (PC) y generación de señales de evento.

\subsection{Protocolo Advanced eXtensible Interface (AXI)}
\label{ssec:protocolo_axi}
% Explicar AXI como estándar de comunicación eficiente (Master/Slave) entre los bloques de hardware.

% -------------------------------------------------------------
% SECCIÓN 4: El Core Técnico del Reporter
% -------------------------------------------------------------
\section{Técnicas de Trazabilidad y Asignación Temporal (Timestamping)}
\label{sec:timestamping}

Este es el segmento más importante, ya que sustenta el mecanismo propuesto por tu Módulo Reporter.

\subsection{Mecanismos de Trazado de Eventos (Event Tracing)}
\label{ssec:event_tracing}
% Definición de qué constituye un "evento" de CPU (instrucción, interrupción, acceso).

\subsection{Generación de Marcas Temporales (Timestamps) en Hardware}
\label{ssec:generacion_timestamps}
% Cómo se genera un timestamp (contadores de clock).
% Análisis de la precisión y tolerancia requerida para STR.

\subsection{Problemas de Sincronización entre Dominios de Clock}
\label{ssec:sincronizacion_clock}
% Discusión sobre cómo asegurar la coherencia temporal al transferir datos entre el dominio del soft-core (PL) y el dominio del ARM (PS).
% Esto sustenta la necesidad de la "señal de referencia" de la hipótesis.

% -------------------------------------------------------------
% SECCIÓN 5: Antecedentes (Opcional, pero recomendable)
% -------------------------------------------------------------
\section{Trabajos Relacionados y Antecedentes}
\label{sec:trabajos_relacionados}

\subsection{Herramientas de Debugging y Trazado Existentes}
\label{ssec:herramientas_existentes}
% Breve revisión crítica de soluciones comerciales o académicas similares, y cómo tu Módulo Reporter se diferencia/mejora estas soluciones.

%%% Local Variables: 
%%% mode: latex
%%% TeX-master: "main"
%%% End: 
