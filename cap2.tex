\chapter{Marco teorico}
\label{ch:marco_teorico}
\graphicspath{{figs/}}



% -------------------------------------------------------------
% SECCIÓN 1: Fundamentos del Problema
% -------------------------------------------------------------
\section{Sistemas de Tiempo Real y Trazabilidad}
\label{sec:sistemas_tiempo_real}

Este apartado establece el contexto y la necesidad de la investigación, justificando por qué la medición temporal es crítica en el campo de aplicación. Este apartado lo voy a llenar con Charlie, dado que tambien quiero tener mas información de como fue impulsado el trabajo.

\subsection{Definición de Sistemas de Tiempo Real (STR)}
\label{ssec:definicion_str}

Un \textbf{Sistema de Tiempo Real (STR)} es aquel cuya corrección no depende únicamente de la precisión lógica del resultado de un cómputo, sino también del \textbf{momento exacto en que dicho resultado se produce}. En estos sistemas, el cumplimiento de las restricciones temporales (\textit{deadlines}) es un factor crucial. La gravedad del fallo al no cumplir con estas restricciones da lugar a dos clasificaciones principales:

\begin{itemize}
    \item \textbf{Sistema de Tiempo Real Estricto (\textit{Hard Real-Time}):} En estos sistemas, las restricciones temporales son de máxima prioridad. El \textbf{incumplimiento de una fecha límite se considera un fallo} del sistema. Un retraso, por mínimo que sea, puede tener consecuencias de fiabilidad.
    
    \item \textbf{Sistema de Tiempo Real Flexible (\textit{Soft Real-Time}):} El incumplimiento ocasional de una fecha límite es indeseable y degrada el rendimiento o la calidad del servicio, pero \textbf{no conduce a un fallo total} del sistema. En este caso, el valor del resultado disminuye después de su \textit{deadline}, pero el sistema puede tolerar las violaciones ocasionales.
\end{itemize}
\cite{liu2006real}




Dado que el objetivo del presente proyecto es evaluar el desempeño de un programa con la máxima precisión posible, el mecanismo de trazabilidad desarrollado debe aspirar a la fiabilidad de un entorno \textit{Hard Real-Time}. Esto requiere garantizar la \textbf{predictibilidad} y la \textbf{mínima intrusión} de la herramienta de medición.
% Definición, tipos (hard, soft), y la importancia del determinismo.

\subsection{Necesidad de la Trazabilidad Temporal (\textit{Tracing}) y el Problema de la Intrusión}
\label{ssec:problema_latencia}

La evaluación del desempeño de cualquier STR requiere medir con precisión los \textbf{tiempos de respuesta} y los \textbf{intervalos de tiempo} entre eventos de interés del sistema. Este proceso de medición se conoce como \textbf{trazabilidad} o \textit{tracing}, y consiste en registrar la secuencia de eventos del programa junto con una marca temporal (\textit{timestamp}) asociada a cada uno.

El principal desafío metodológico en la trazabilidad de STR es el \textbf{Efecto Sonda} o \textbf{Intrusión de la Medición} (\textit{probe effect}). Este fenómeno ocurre cuando la herramienta utilizada para la medición afecta el comportamiento temporal del sistema que se está observando.

Un ejemplo de esta limitación se evidenció en proyectos antecesor que empleaban técnicas de \textit{debugging} para la trazabilidad. Estos enfoques se basaban en la interrupción del tiempo de ejecución del procesador para generar las marcas temporales, resultando en un método altamente intrusivo y, por ende, \textbf{no fiable} para la medición de sistemas deterministas.

Las técnicas de trazabilidad basadas en \textbf{software} (como la inserción de llamadas a funciones de registro o contadores en el código del programa) son inherentemente intrusivas debido a dos razones principales:
\begin{enumerate}
    \item \textbf{Alteración de la Temporización:} La ejecución del código de registro consume tiempo de CPU y recursos de memoria, añadiendo una latencia variable que altera la secuencia temporal original del programa.
    \item \textbf{Distorsión del Contexto:} El código de instrumentación puede modificar el uso de la caché o la jerarquía de memoria, lo que cambia las condiciones bajo las cuales el programa se ejecutaría normalmente.
\end{enumerate}

Para sistemas que aspiran a la predictibilidad de un entorno \textit{Hard Real-Time}, esta alteración es inaceptable, ya que invalida los resultados de la medición.

Por lo tanto, la solución propuesta en este proyecto es el desarrollo del \textbf{Módulo Reporter}, una herramienta de \textbf{trazabilidad basada en hardware} implementada en la Lógica Programable (PL) de la FPGA. Al operar de manera paralela y asíncrona al procesador bajo prueba (\textit{soft-core}), el Módulo Reporter minimiza la intrusión y se espera obtener \textit{timestamps} de alta fidelidad, superando las limitaciones impuestas por las técnicas de instrumentación basadas en software.

% -------------------------------------------------------------
% SECCIÓN 2: Plataforma de Hardware: FPGA y Arquitectura Zynq
% -------------------------------------------------------------
\section{Arquitectura Heterogénea en Sistemas de Lógica Programable}
\label{sec:arquitectura_fpga}

Se describe la plataforma física sobre la que se implementará la solución, detallando la interconexión PS-PL.

\subsection{Fundamentos de Field-Programmable Gate Arrays (FPGA)}
\label{ssec:fundamentos_fpga}
% Brevemente la tecnología: Lógica Programable (PL) vs. Processing System (PS).

\subsection{Arquitectura System-on-Chip (SoC) Zynq}
\label{ssec:arquitectura_zynq}
% Detalle de la interconexión PS-PL, el ARM embebido y su rol como plataforma anfitriona.
% Incluir la Figura 1 (esquema general de la placa) si no se incluyó en la Introducción.

\subsection{Uso de Linux en el Processing System (PS)}
\label{ssec:linux_ps}
% Justificación del uso de Linux para recepción, procesamiento y almacenamiento de datos.

% -------------------------------------------------------------
% SECCIÓN 3: La Unidad Bajo Prueba y la Interconexión
% -------------------------------------------------------------
\section{Procesadores Soft-Core y Protocolo de Comunicación}
\label{sec:softcore_comunicacion}

Se centra en la unidad que genera los eventos y el medio por el cual se transfieren.

\subsection{Procesador Soft-Core RISC-V}
\label{ssec:riscv}
% Introducción al ISA RISC-V.
% Requisitos para la síntesis en PL: acceso a registros (PC) y generación de señales de evento.

\subsection{Protocolo Advanced eXtensible Interface (AXI)}
\label{ssec:protocolo_axi}
% Explicar AXI como estándar de comunicación eficiente (Master/Slave) entre los bloques de hardware.

% -------------------------------------------------------------
% SECCIÓN 4: El Core Técnico del Reporter
% -------------------------------------------------------------
\section{Técnicas de Trazabilidad y Asignación Temporal (Timestamping)}
\label{sec:timestamping}

Este es el segmento más importante, ya que sustenta el mecanismo propuesto por tu Módulo Reporter.

\subsection{Mecanismos de Trazado de Eventos (Event Tracing)}
\label{ssec:event_tracing}
% Definición de qué constituye un "evento" de CPU (instrucción, interrupción, acceso).

\subsection{Generación de Marcas Temporales (Timestamps) en Hardware}
\label{ssec:generacion_timestamps}
% Cómo se genera un timestamp (contadores de clock).
% Análisis de la precisión y tolerancia requerida para STR.

\subsection{Problemas de Sincronización entre Dominios de Clock}
\label{ssec:sincronizacion_clock}
% Discusión sobre cómo asegurar la coherencia temporal al transferir datos entre el dominio del soft-core (PL) y el dominio del ARM (PS).
% Esto sustenta la necesidad de la "señal de referencia" de la hipótesis.

% -------------------------------------------------------------
% SECCIÓN 5: Antecedentes (Opcional, pero recomendable)
% -------------------------------------------------------------
\section{Trabajos Relacionados y Antecedentes}
\label{sec:trabajos_relacionados}

\subsection{Herramientas de Debugging y Trazado Existentes}
\label{ssec:herramientas_existentes}
% Breve revisión crítica de soluciones comerciales o académicas similares, y cómo tu Módulo Reporter se diferencia/mejora estas soluciones.

%%% Local Variables: 
%%% mode: latex
%%% TeX-master: "main"
%%% End: 
